\chapter{Elasticities}

\section{The Elasticity of Demand}

\subsection{The Price Elasticity of Demand}

\begin{itemize}

	\item A good's \underline{price elasticity of demand} measures how much the quantity demanded responds to a change in price.
	
	\item Demand for a good is \underline{elastic} if the quantity demanded responds a lot to changes in price.
	
	\item Demand is \underline{inelastic} if the quantity demanded doesn't respond a lot to changes in price. 

\end{itemize}

\underline{Determinants of Price Elasticity of Demand}

\begin{enumerate}

	\item Availability of Substitutes: Goods with close substitutes are more elastic
		
	\item Necessities v. Luxuries: Necessities are more inelastic and luxuries are more elastic
	
	\item Market Definition: Goods in narrowly defined markets are more elastic, and goods in broadly defined markets are more inelastic.
	
		\begin{itemize}
		
		\item Ex. Vanilla ice cream is a narrowly defined good, so it has lots of substitutes.
		
		\item Ex. Food is a broadly defined good, so it has no substitutes.
		
		\end{itemize}
		
	\item Time Horizon: Goods are more elastic in the long term than the short term
	
%		\begin{itemize}
%		
%		\item Ex. In the short run, demand for gasoline is inelastic because most people can't readily change their commute.
%		
%		\item Ex. In the long run, demand for gasoline is elastic because people buy more fuel-efficient cars, switch to public transportation, or move closer to work. 
%		
%		\end{itemize}

\end{enumerate}



\subsection{Computing the Price Elasticity of Demand}



\subsection{The Midpoint Method}