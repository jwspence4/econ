\documentclass{article}

\title{Supply Problem Set Solutions}
\author{Principles of Microeconomics}
\date{\today}

\begin{document}

\maketitle

\begin{enumerate}

\item (a) -- movements along the supply curve come from changes in the price of the good.

\item (c) -- a decrease in the price of an input makes producing pizza more profitable, so supply shifts right.

\item (d) -- Tricky tricky, Mr. Spence. The price of a substitute affects demand, not supply. When the price of a substitute increases, demand increases.

\item Shift; movement along.

\item The supply curve shifts left because there is a negative shock to the production "technology" (agriculture), so farms can't produce as much. 

\item

	\begin{enumerate}
	
	\item Supply decreases because the price of an input increases.
	
	\item Supply increases because the production technology improves. 
	
	\end{enumerate}
	
\item

	\begin{enumerate}
	
	\item Supply decreases because the number of sellers (or really the quantity that each seller can sell) decreases. 
	
	\item Supply increases because the production technology improves. 
	
	\end{enumerate}

\end{enumerate}

\end{document}