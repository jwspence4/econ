\documentclass{article}
\usepackage{amsmath}

\title{Elasticity of Supply and Demand Problem Set}
\author{Principles of Microeconomics}
\date{\today}

\begin{document}

\maketitle

\subsection*{Instructions} Please answer on a separate sheet of paper and submit on Gradescope.

\subsection*{Multiple Choice} Please briefly explain your choice with a sentence or calculation.

\begin{enumerate}

\item The price of a good rises from \$16 to \$24, and the quantity supplied rises from 90 to 110 units. Calculated with the midpoint method, the price elasticity of supply is

	\begin{enumerate}
	
	\item 0.2.
	
	\item 0.5.
	
	\item 2.
	
	\item 5.	
	
	\end{enumerate}
	
\item If the price elasticity of supply is zero, the supply curve is

	\begin{enumerate}
	
	\item upward sloping.
	
	\item horizontal.
	
	\item vertical.
	
	\item fairly flat at low quantities but steeper at larger quantities.
	
	\end{enumerate}
	
\item The ability of firms to enter and exit a market over time means that, in the long run,

	\begin{enumerate}
	
	\item the demand curve is more elastic.
	
	\item the demand curve is less elastic.
	
	\item the supply curve is more elastic.
	
	\item the supply curve is less elastic.
	
	\end{enumerate}

\end{enumerate}

\subsection*{Free Response}

\begin{enumerate}
\setcounter{enumi}{3}

\item The price of aspirin rose sharply last month, while the quantity sold remained the same. Five people suggest various diagnoses of the phenomenon:

	\begin{tabular}{| c | c |}
	\hline
	Meredith: & Demand increased, but supply was perfectly inelastic. \\
	\hline
	Alex: & Demand increased, but it was perfectly inelastic. \\
	\hline
	Miranda: & Demand increased, but supply decreased at the same time. \\
	\hline
	Richard: & Supply decreased, but demand was unit elastic. \\
	\hline
	Owen: & Supply decreased, but demand was perfectly inelastic. \\
	\hline
	\end{tabular}
	
	Who could possibly be right? Please explain with a graph for each person you think could be right.
	
\item The \textit{New York Times} reported (Feb. 17, 1996) that subway ridership declined after a fare increase: ``There were nearly four million fewer riders in December 1995, the first full month after the price of a token increased 25 cents to \$1.50, than in the previous December, a 4.3 percent decline."

	\begin{enumerate}
	
	\item Use these data to estimate the price elasticity of demand for subway rides. Please show all relevant calculations.
	
	\item According to your estimate, what happens to the Transit Authority's revenue when the fare rises? Please explain in a sentence.
	
	\item Why might your estimate of the elasticity be unreliable? Please explain in a sentence.
	
	\end{enumerate}
	
\item You are the curator of a museum. The museum is running short of funds, so you would like to increase revenue. Should you increase or decrease the price of admission? Please explain in two sentences.

\item How long did this problem set take you?

%\item for this PS -- 6, 8, 11
%
%\item Done: 1, 2, 3, 4, 7, 9
%
%\item Need: 5, 10, 12

\end{enumerate}

\end{document}