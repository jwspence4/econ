\documentclass{article}
\usepackage{amsmath}

\title{Elasticity of Supply}
\author{Principles of Microeconomics}
\date{\today}

\begin{document}

\maketitle

\subsection*{Instructions} Please answer on a separate sheet of paper and submit on Gradescope.

\subsection*{Multiple Choice} Please briefly explain your choice with a sentence or calculation.

\begin{enumerate}

\item The price of a good rises from \$16 to \$24, and the quantity supplied rises from 90 to 110 units. Calculated with the midpoint method, the price elasticity of supply is

	\begin{enumerate}
	
	\item 0.2.
	
	\item 0.5.
	
	\item 2.
	
	\item 5.	
	
	\end{enumerate}
	
\item If the price elasticity of supply is zero, the supply curve is

	\begin{enumerate}
	
	\item upward sloping.
	
	\item horizontal.
	
	\item vertical.
	
	\item fairly flat at low quantities but steeper at larger quantities.
	
	\end{enumerate}
	
\item The ability of firms to enter and exit a market over time means that, in the long run,

	\begin{enumerate}
	
	\item the demand curve is more elastic.
	
	\item the demand curve is less elastic.
	
	\item the supply curve is more elastic.
	
	\item the supply curve is less elastic.
	
	\end{enumerate}

\end{enumerate}

\subsection*{Free Response}

\begin{enumerate}
\setcounter{enumi}{3}

\item 6

\end{enumerate}

\end{document}