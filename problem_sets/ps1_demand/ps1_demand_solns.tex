\documentclass{article}

\title{Demand Problem Set Solutions}
\author{Principles of Microeconomics}
\date{\today}

\begin{document}

\maketitle

\begin{enumerate}

\item (b) -- A change in the price of hamburgers leads to a movement along the demand curve.

\item (a) -- When the price of a substitute goes up, the demand for pizza increases.

\item (d) -- Demand for an inferior good falls when incomes rise.

\item Shift; movement along.

\item Pumpkin juice is an inferior good because his demand for it rises as his income falls. His demand curve shifts right because he buys more. 

\item It shifts left (decreases) because people's tastes shift towards a substitute.

\item 

	\begin{enumerate}
	
	\item Demand increases because people's taste for minivans will increase.
	
	\item Demand increases because the price of a substitute increases.
	
	\item Demand decreases because cars are normal goods and people's income has declined. 
	
	\end{enumerate}
	
\item

	\begin{enumerate}
	
	\item Complements
	
	\item Substitutes
	
	\item Substitutes
	
	\end{enumerate}
	
\item

	\begin{enumerate}
	
	\item Demand decreases because the price of a substitute falls.
	
	\item Demand increases because the number of buyers increases.
	
	\end{enumerate}

\end{enumerate}

\end{document}