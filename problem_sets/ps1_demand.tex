\documentclass{article}

\title{Demand Problem Set}
\author{Principles of Microeconomics}
\date{\today}

\begin{document}

\maketitle

\subsection*{Instructions} Please answer on a separate sheet of paper and submit on Gradescope. 

\subsection*{Multiple Choice} Please briefly explain your choice. One sentence is sufficient.

\begin{enumerate}

\item A change in which of the following will not shift the demand curve for hamburgers?

	\begin{enumerate}
	
	\item the price of hot dogs
	
	\item the price of hamburgers
	
	\item the price of hamburger buns
	
	\item the income of hamburger consumers
	
	\end{enumerate}
	
\item Which of the following will shift the demand curve for pizza to the right?

	\begin{enumerate}
	
	\item an increase in the price of hamburgers, a substitute for pizza
	
	\item an increase in the price of root beer, a complement to pizza
	
	\item the departure of college students, as they leave for summer vacation
	
	\item a decrease in the price of pizza
	
	\end{enumerate}
	
\item If pasta is an inferior good, then the demand curve shifts to the \underline{\phantom{xxxxxxxxxx}} when \underline{\phantom{xxxxxxxxxx}} rises.

	\begin{enumerate}
	
	\item right; the price of pasta
	
	\item right; consumers' income
	
	\item left; the price of pasta
	
	\item left; consumers' income
	
	\end{enumerate}

\end{enumerate}

\subsection*{Free Response}

\begin{enumerate}
\setcounter{enumi}{3}

\item Does a change in consumers' tastes lead to a movement along the demand curve or to a shift in the demand curve? Does a change in price lead to a movement along the demand curve or to a shift in the demand curve? No explanation necessary. 

\item Harry's income declines and as a result, he buys more pumpkin juice. Is pumpkin juice an inferior or a normal good? What happens to Harry's demand curve for pumpkin juice? Two sentences are sufficient.

\item When the weather turns warm in New England every summer, what happens to the demand curve for hotel rooms in Caribbean resorts? One sentence is sufficient.

\item Consider the market for minivans. For each of the events listed here, explain whether demand increases or decreases. One sentence for each is sufficient.

	\begin{enumerate}
	
	\item People decide to have more children.
	
	\item The price of sports utility vehicles rises.
	
	\item A stock market crash lowers people's income.
	
	\end{enumerate}
	
\item For each pair, identify whether they are complements or substitutes:

	\begin{enumerate}
	
	\item Film streaming and TV screens
	
	\item Film streaming and movie tickets
	
	\item TV screens and movie tickets
	
	\end{enumerate}
	
\item What effect does each of the following have on demand for sweatshirts? One sentence for each is sufficient. 

	\begin{enumerate}
	
	\item The price of leather jackets falls.
	
	\item All colleges require morning exercise in appropriate attire.
	
	\end{enumerate}

\end{enumerate}

\end{document}


%Demand ---
% QQ #4-6
% QfR #3 (no explanation), 4
% P&A #1b, 3ade, 4a, 6bc,

%Supply ---
% QQ #7-9
% QfR #6
% P&A #1ac, 3bc, 4b, 6ad

%Demand and Supply ----