\documentclass{article}
\usepackage{makecell}

\title{Elasticities of Demand Problem Set 1}
\author{Principles of Microeconomics}
\date{\today}

\begin{document}

\maketitle

\subsection*{Instructions} Please answer on a separate sheet of paper and submit on Gradescope. 

\subsection*{Multiple Choice} Please briefly explain your choice with a sentence.

\begin{enumerate}

\item A good tends to have inelastic demand if

	\begin{enumerate}
	
	\item the good is a necessity.
	
	\item there are many close substitutes.
	
	\item the market is narrowly defined.
	
	\item the long-run response is being measured.
	
	\end{enumerate}
	
\item The price of a good rises from \$8 to \$12, and the quantity demanded falls from 110 to 90 units. Calculated with the midpoint method, the elasticity is

	\begin{enumerate}
	
	\item 1/5.
	
	\item 1/2.
	
	\item 2.
	
	\item 5.
	
	\end{enumerate}

\end{enumerate}

\subsection*{Free Response}

\begin{enumerate}

\item For each of the following pairs of goods, which good would you expect to have more elastic demand? Please explain in a sentence.

	\begin{enumerate}
	
	\item required textbooks or mystery novels
	
	\item Adele recordings or pop music recordings in general
	
	\item subway rides during the next six months or subway rides during the next five years
	
	\end{enumerate}

\item Suppose that business travelers and vacationers have the following demand for airline tickets from Chicago to Miami:

	\begin{center}
	\begin{tabular}{c | c | c}
	\textbf{Price} & \makecell{\textbf{Quantity Demanded} \\ \textbf{(business travelers)}} & \makecell{\textbf{Quantity Demanded} \\ \textbf{(vacationers)}} \\
	\hline
	\$150 & 2,100 tickets & 1,000 tickets \\
	200 & 2,000 & 800 \\
	250 & 1,900 & 600 \\
	300 & 1,800 & 400 
	\end{tabular}
	\end{center}
	
	\begin{enumerate}
	
	\item As the price of tickets rises from \$200 to \$250, what is the price elasticity of demand for (i) business travelers and (ii) vacationers? (Use the midpoint method in your calculations.)
	
	\item Why might vacationers and business travelers have different elasticities? Please explain in a sentence.
	
	\end{enumerate}

\item Suppose the price elasticity of demand for heating oil is 0.2 in the short run and 0.7 in the long run.

	\begin{enumerate}
	
	\item If the price of heating oil rises from \$1.80 to \$2.20 per gallon, what happens to the quantity of heating oil demanded in the short run? In the long run? (Use the midpoint method in your calculations.)
	
	\item Why might this elasticity depend on the time horizon? Please explain in a sentence.
	
	\end{enumerate}

\end{enumerate}

\end{document}