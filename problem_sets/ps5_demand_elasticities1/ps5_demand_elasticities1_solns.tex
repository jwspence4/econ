\documentclass{article}
\usepackage{amsmath}

\title{Elasticities of Demand Problem Set 1 Solutions}
\author{Principles of Microeconomics}
\date{\today}

\begin{document}

\maketitle

\begin{enumerate}

\item (a) -- If a good is a necessity, that means there aren't many if any substitutes, so consumers can't reduce their demand much in response to a price increase. 

\item (b)

	\begin{gather*}
	\% \Delta Q = \frac{90 - 110}{\frac{90 + 110}{2}} = \frac{-20}{100} = -\frac{1}{5} \\
	\% \Delta P = \frac{12 - 8}{\frac{12 + 8}{2}} = \frac{4}{10} = \frac{2}{5} \\
	\eta = \left| \frac{\% \Delta Q}{\% \Delta P} \right| = \frac{\frac{1}{5}}{\frac{2}{5}} = \frac{1}{2}
	\end{gather*}

\item

	\begin{enumerate}
	
	\item I'd expect mystery novels to be more elastic because they are more of a luxury while required textbooks are more of a necessity.
	
	\item I'd expect Adele recordings to be more elastic because they are a more narrowly defined good, so it's easier to find a substitute for Adele than it is pop music as a whole.
	
	\item I'd expect subway rides during the next five years to be more elastic because the longer time horizon gives people more time to change their behavior and find substitutes.
	
	\end{enumerate}
	
\item

	\begin{enumerate}
	
	\item 
	
		\begin{enumerate}
		
		\item 
		\begin{gather*}
		\% \Delta Q = \frac{1900 - 2000}{\frac{1900 + 2000}{2}} = \frac{-100}{1950} = -\frac{2}{39} \\
		\% \Delta P = \frac{250 - 200}{\frac{250 + 200}{2}}  = \frac{50}{225} = \frac{2}{9} \\
		\eta = \left| \frac{\% \Delta Q}{\% \Delta P} \right| = \frac{\frac{2}{39}}{\frac{2}{9}} = \frac{9}{39} \approx 0.23
		\end{gather*}
		
		\item
		\begin{gather*}
		\% \Delta Q = \frac{600 - 800}{\frac{600 + 800}{2}} = \frac{-200}{700} = \frac{-2}{7} \\
		\% \Delta P = \frac{2}{9} \text{ (See above)} \\
		\eta = \left| \frac{\% \Delta Q}{\% \Delta P} \right| = \frac{\frac{2}{7}}{\frac{2}{9}} = \frac{9}{7} \approx 1.29
		\end{gather*}
		
		\end{enumerate}
		
	\item Flights are more of a necessity for business travelers and more of a luxury for vacationers, so we would expect vacationers to have more elastic demand.
	
	\end{enumerate}
	
\item

	\begin{enumerate}
	
	\item Short run:
	\begin{align*}
	\% \Delta P &= \frac{2.2 - 1.8}{\frac{2.2 + 1.8}{2}} = \frac{0.4}{2} = 0.2\% \\
	\eta_{SR} &= \left| \frac{\% \Delta Q_{SR}}{\% \Delta P} \right| \\
	0.2 &= \frac{| \% \Delta Q_{SR} |}{0.2} \\
	0.04 &= | \% \Delta Q_{SR} | \\
	-0.04 &= \% \Delta Q_{SR} \text{ (by the law of demand)}
	\end{align*}
	In the short run, the quantity of heating oil demanded decreases by 0.04\%.
	
	\vspace{5mm}
	
	Long run:
	\begin{align*}
	\eta_{LR} &= \left| \frac{\% \Delta Q_{LR}}{\% \Delta P} \right| \\
	0.7 &= \frac{| \% \Delta Q_{LR} |}{0.2} \\
	0.14 &= | \% \Delta Q_{LR} | \\
	-0.14 &= \% \Delta Q_{LR} \text{ (by the law of demand)}
	\end{align*}
	In the long run, the quantity of heating oil demanded decreases by 0.14\%.
	
	\item A longer time horizon gives people more time to find substitutes for heating oil and change their behavior -- for example switching their home to natural gas -- so we would expect demand to be more elastic in the long run.
	
	\end{enumerate}

\end{enumerate}

\end{document}