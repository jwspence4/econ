\documentclass{article}
\usepackage{makecell}

\title{Elasticities of Demand Problem Set 2}
\author{Principles of Microeconomics}
\date{\today}

\begin{document}

\maketitle

\subsection*{Instructions} Please answer on a separate sheet of paper and submit on Gradescope.

\subsection*{Multiple Choice} Please briefly explain your choice with a sentence.

\begin{enumerate}

\item An increase in a good's price reduces the total amount consumers spend on the good if the \underline{\phantom{xxxxxxxxxx}} elasticity of demand is \underline{\phantom{xxxxxxxxxx}} than one.

	\begin{enumerate}
	
	\item income; less
	
	\item income; greater
	
	\item price; less
	
	\item price; greater
	
	\end{enumerate}
	
\item A linear downward-sloping demand curve is

	\begin{enumerate}
	
	\item inelastic.
	
	\item unit elastic.
	
	\item elastic.
	
	\item inelastic at some points and elastic at others.
	
	\end{enumerate}
	
\item The citizens of Lilliput spend a higher fraction of their income on food than do the citizens of Brobdingnag. The reason could be that
	
	\begin{enumerate}
	
	\item Lilliput has lower food prices, and the price elasticity of demand is zero.
	
	\item Lilliput has lower food prices, and the price elasticity of demand is 0.5. 
	
	\item Lilliput has lower income, and the income elasticity of demand is 0.5.
	
	\item Lilliput has lower income, and the income elasticity of demand is 1.5.
	
	\end{enumerate}

\end{enumerate}

\subsection*{Free Response}

\begin{enumerate}
\setcounter{enumi}{3}

\item A price change causes the quantity demanded of a good to decrease by 30\% while the total revenue of that good increases by 15\%. Is the demand curve elastic or inelastic? Please explain in a sentence. 

\item Two drivers, Walt and Jessie, each drive up to a gas station. Before looking at the price, each places an order. Walt says, ``I'd like 10 gallons of gas." Jessie says, ``I'd like \$10 worth of gas." What is each driver's price elasticity of demand? Please explain in two sentences.
	
\item Suppose that your demand schedule for pizza is as follows:

	\begin{center}
	\begin{tabular}{c | c | c}
	\textbf{Price} & \makecell{\textbf{Quantity Demanded} \\ \textbf{income = \$20,000}} & \makecell{\textbf{Quantity Demanded} \\ \textbf{income = \$24,000}} \\
	\hline
	\$8 & 40 pizzas & 50 pizzas \\
	10 & 32 & 45 \\
	12 & 24 & 30 \\
	14 & 16 & 20 \\
	16 & 8 & 12
	\end{tabular}
	\end{center}
	
	\begin{enumerate}
	
	\item Use the midpoint method to calculate your price elasticity of demand as the price of pizza increases from \$8 to \$10 if (i) your income is \$ 20,000 and (ii) your income is \$24,000. Please show all relevant calculations.
	
	\item Use the midpoint method to calculate your income elasticity of demand as your income increases from \$20,000 to \$24,000 if (i) the price is \$12 and (ii) the price is \$16.
	
	\end{enumerate}
	
%\item Cups of coffee and donuts are complements. Both have inelastic demand. A hurricane destroys half the coffee bean crop.
%
%	\begin{enumerate}
%	
%	\item What happens to the price of coffee beans? Please explain using a diagram and a summarizing sentence.
%	
%	\item What happens to the price of a cup of coffee? Please explain using a diagram and a summarizing sentence.
%	
%	\item What happens to total expenditure on cups of coffee? Please explain in a sentence. 
%	
%	\item What happens to the price of donuts? Please explain using a diagram and a summarizing sentence.
%	
%	\item What happens to total expenditure on donuts? Please explain in a sentence.
%	
%	\end{enumerate}
	
\item How long did this problem set take you?

\end{enumerate}

\end{document}