\documentclass{article}
\usepackage{graphicx, amsmath}
\graphicspath{ {./images/} }

\title{Elasticities of Demand Problem Set 2 Solutions}
\author{Principles of Microeconomics}
\date{\today}



\begin{document}

\maketitle

\begin{enumerate}

\item (d) -- An increase in price leads to a decrease in total revenue if demand is elastic (with respect to price).

\item (d) -- The elasticity of a linear demand curve is non-constant.

\item (c) -- If the income elasticity of demand is 0.5, the decrease in income between Brobdingnag and Lilliput leads to a proportionately smaller decrease in $Q_D$. Since Lilliput's income declines more than its quantity demanded relative to Brobdingnag, Lilliput must spend a higher fraction of its income on food. 

\item The demand curve is inelastic because if $Q_D$ decreases that means $P$ increases (by the law of demand). If total revenue increases in response to an increase in $P$, demand must be inelastic.

\item Walt's demand is perfectly inelastic ($\eta = 0$) because he demands the same quantity regardless of price. Jessie's demand is unit elastic ($\eta = 1$) because he spends the same amount regardless of price. That means  the revenue generated by sales to Jessie don't change with price. That's only possible if Jessie's demand is unit elastic. 

\item

	\begin{enumerate}
	
	\item
	
		\begin{enumerate}
		
		\item 
		\begin{gather*}
		\% \Delta Q = \frac{32 - 40}{\frac{32 + 40}{2}} = -\frac{8}{36} = -\frac{2}{9} \\
		\% \Delta P = \frac{10 - 8}{\frac{10 + 8}{2}} = \frac{2}{9} \\
		\eta = \left| \frac{\% \Delta Q}{\% \Delta P} \right| = 1
		\end{gather*}
		
		\item
		\begin{gather*}
		\% \Delta Q = \frac{45 - 50}{\frac{45 + 50}{2}} = \frac{-5}{47.5} = \frac{-10}{95} \\
		\% \Delta P = \frac{2}{9} \\
		\eta = \left| \frac{\% \Delta Q}{\% \Delta P} \right| = \frac{90}{190} = \frac{9}{19} \approx 0.47
		\end{gather*}
		
		\end{enumerate}
	
	\item
	
		\begin{enumerate}
		
		\item
		\begin{gather*}
		\% \Delta Q = \frac{30 - 24}{\frac{30 + 24}{2}} = \frac{6}{27} = \frac{2}{9} \\
		\% \Delta I = \frac{24,000 - 20,000}{\frac{24,000 + 20,000}{2}} = \frac{4,000}{22,000} = \frac{2}{11} \\
		IED = \frac{\% \Delta Q}{\% \Delta I} = \frac{\frac{2}{9}}{\frac{2}{11}} = \frac{11}{9} \approx = 1.22
		\end{gather*}
		
		\item
		\begin{gather*}
		\% \Delta Q = \frac{12 - 8}{\frac{12 + 8}{2}} = \frac{4}{10} = \frac{2}{5} \\
		\% \Delta I = \frac{2}{11} \\
		IED = \frac{\% \Delta Q}{\% \Delta I} = \frac{\frac{2}{5}}{\frac{2}{11}} = \frac{11}{5} = 2.2
		\end{gather*}
		
		\end{enumerate}
	
	\end{enumerate}

\end{enumerate}

\end{document}